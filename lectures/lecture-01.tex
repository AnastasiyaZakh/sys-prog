\subsection{Мови програмування}

При вивченні мов програмування, як правило, виділяють три аспекти:
\begin{itemize}
	\item Прагматичний;
	\item Семантичний;
	\item Синтаксичний.
\end{itemize}

\subsubsection{Прагматичний аспект}

\textit{Прагматичний аспект} (прагматика мови програмування) визначає клас задач, на
розв'язування яких орієнтується мова програмування. Як правило, прагматичний аспект
менш формалізований у порівнянні з семантичним та синтаксичним аспектами. \medskip

За класом задач на розв'язування яких орієнтуються мови програмування 
їх можна поділити передусім на 
\begin{itemize}
	\item процедурні;
	\item непроцедурні.
\end{itemize}

\textit{Процедурні} мови програмування орієнтовані перш за все на опис
(визначення) алгоритмів, тобто по суті використовуються для побудови процедур
обробки даних. До таких мов ми відносимо всім відомі мови програмування, такі
як Pascal, Fortran, C та ін. \medskip

\textit{Непроцедурні} мови програмування на відміну від процедурних неявно
визначають процедури обробки даних. Частіше всього такі мови
використовуються для побудови завдань на обробку даних. При цьому, за
допомогою інструкцій непроцедурної мови програмування визначається що
необхідно зробити з даними і явно не визначається як (з використанням яких
алгоритмів) необхідно розв'язати задачу. До непроцедурних мов програмування
ми відносимо командні мови операційних систем, мови управління в пакетах
прикладних програм та ін. \medskip

Як процедурні, так і непроцедурні мови програмування можуть
орієнтуватися як на декілька класів задач, так і конкретну предметну область. У
першому випадку ми будемо говорити про \textit{універсальні} мови програмування
(Pascal, Fortran, C), в другому --- про \textit{спеціалізовані} мови програмування (Snobol, Lisp).

\subsubsection{Семантичний аспект}

\textit{Семантичний аспект} (семантика мови програмування) визначається
шляхом конкретизації базових функцій обробки даних, набору конструкцій
управління та методами побудови більш ``складних'' програм на основі ``простих''. \medskip

Наприклад, визначивши як базовий тип даних ``рядок'' ми повинні
запропонувати ``традиційний'' набір функцій обробки таких даних: порівняння
рядків, виділення частини рядка, конкатенацію рядків та ін.  \medskip

Семантика мови програмування має бути визначена формально, бо інакше у
подальшому неможливо буде побудувати відповідний мовний процесор. Станом на
сьогодні існують два основних напрямки визначення семантики мов
програмування: 
\begin{itemize}
	\item методи денотаційної семантики;
	\item методи операційної семантики.
\end{itemize}

Методи \textit{денотаційної семантики} базуються на відповідних алгебрах, методи
\textit{операційної семантики} базуються на синтаксичних структурах програм.

\subsubsection{Синтаксичний аспект}

\textit{Синтаксичний аспект} (синтаксис мови програмування) визначає набір
синтаксичних конструкцій мови програмування, які використовуються для нотації
(запису) семантичних одиниць в програмі. Про синтаксис мови програмування
можна сказати як про форму, яка є суть похідною від семантики. Для визначення
(опису) синтаксису мови програмування використовуються як механізми, що
орієнтовані на синтез, так і механізми, орієнтовані на аналіз. \medskip

Задачі аналізу та синтезу синтаксичних структур програм --- це дуальні задачі. 
Їх конкретизацію ми будемо розглядати в наступних розділах.  \medskip

Виходячи з вищенасказаного, щоб побудувати мову програмування потрібно:
\begin{itemize}
	\item визначити клас (класи) задач, на розв'язок яких орієнтована мова 
	програмування;
	\item виділити базові типи даних та функції їх обробки, вказати конструкції
	управління в програмах. Побудувати механізми конструювання більш складних
	програм та структур даних на основі більш простих одиниць;
	\item визначити синтаксис мови програмування.
\end{itemize}

\subsection{Мовні процесори}

\textit{Мовні процесори} реалізують мови програмування. Точніше, мовний
процесор призначений для обробки програм відповідної мови програмування. З
точки зору прагматики, мовні процесори діляться на
\begin{itemize}
	\item транслятори;
	\item інтерпретатори.
\end{itemize}

\textit{Мовний процесор типу транслятор (транслятор)} --- це програмний
комплекс, котрий на вході отримує текст програми на вхідній мові, а на виході
видає версію програми на вихідній мові, що називається об'єктною мовою. В
більшості випадків як об'єктна мова виступає мова команд деякої обчислювальної
машини. Серед трансляторів можна виділити дві програмні системи:
\begin{itemize}
	\item компілятори --- транслятори з мов програмування високого рівня;
	\item асемблери --- транслятори машинно-орієнтованих мов програмування.
\end{itemize}

\textit{Мовний процесор типу інтерпретатор (інтерпретатор)} --- це програмний
комплекс, котрий на вході отримує текст програми на вхідній мові та вхідні дані,
які в подальшому обробляються програмою, а на виході видає результати
обчислень (вихідні дані). \medskip

Оскільки транслятори та інтерпретатори реалізують мови програмування,
вони мають спільні риси: їх структура досить схожа, в основу їх реалізації
покладено спільні теоретичні результати та практичні методи реалізації.

\subsubsection{Структура транслятора}

\begin{enumerate}
	\item Вхідний текст програми
	\item Лексичний аналіз
	\item Синтаксичний аналіз
	\item Семантичний аналіз
	\item Оптимізація проміжного коду
	\item Генерація коду
	\item Вихідний (об'єктний) код
\end{enumerate}

\subsubsection{Призначення основних компонентів транслятора}

\begin{enumerate}
	\item \textit{Лексичний аналізатор.} \medskip
	
	\textbf{Вхід:} вхідний текст (послідовність літер) програми. \medskip

	\textbf{Вихід:} послідовність лексем програми. \medskip

	\textit{Лексема} --- це ланцюжок літер, що має певний зміст. Всі лексеми мови
	програмування (їх кількість, як правило, нескінчена) можна розбити на скінчену
	множину класів. Для більшості мов програмування актуальні наступні класи
	лексем:
	\begin{itemize}
		\item зарезервовані слова;
		\item ідентифікатори;
		\item числові константи (цілі та дійсні числа);
		\item літерні константи;
		\item рядкові константи;
		\item коди операцій;
		\item коментарі. Безпосередньо не несуть інформації щодо структури
		програми. В подальшому не використовуються, тобто не передаються
		синтаксичному аналізатору.
		\item дужки та інші елементи програми.
	\end{itemize}

	\item \textit{Синтаксичний аналізатор.} \medskip

	\textbf{Вхід:} послідовність лексем програми. \medskip
	
	\textbf{Вихід:}
	\begin{itemize}
		\item ``Так'' + синтаксична структура (синтаксичний терм) програми,
		\item ``Ні'' + синтаксичні помилки в програмі.
	\end{itemize}

3. \textit{Семантичний аналізатор.} \medskip

	\textbf{Вхід:} Синтаксичний терм програми. \medskip

	\textbf{Вихід:} \medskip
	\begin{itemize}
		\item ``Так'' + семантична структура (семантичний терм) програми,
		\item ``Ні'' + семантичні помилки в програмі.
	\end{itemize}

4. \textit{Оптимізація проміжного коду.} \medskip

	\textbf{Вхід:} семантичний терм програми. \medskip

	\textbf{Вихід:} оптимізований семантичний терм програми. \medskip

	\textit{Оптимізація} --- це еквівалентне перетворення програми на основі певних
	критеріїв. Серед критеріїв оптимізації можна виділити:
	\begin{itemize}
		\item оптимізацію по пам'яті;
		\item оптимізацію по швидкості виконання. 
	\end{itemize}
	
	В залежності від підходів по оптимізації програми можна розглядати такі
	методи оптимізації:
	\begin{itemize}
		\item машинно-залежні;
		\item машинно-незалежні.
	\end{itemize}

	На відміну від машинно-незалежних
	методів машинно-залежні методи оптимізації враховують архітектурні особливості
	ЕОМ, наприклад, наявність апаратного стека, наявність вільних регістрів, тощо.

	\item \textit{Генерація об'єктного коду.} \medskip

	\textbf{Вхід:} семантичний терм програми. \medskip

	\textbf{Вихід:} результуючий (об'єктний) код програми.
\end{enumerate}

\subsection{Контрольні запитання}

\begin{enumerate}
	\item Які три аспекти як правило виділяють при вивчення мов програмування? 
	% прагматичний, синтаксичний, семантичний
	\item Які два поділи мов програмування в залежності від орієнтації на розв'язання 
	тих чи інших класів задач вам відомі? 
	% процедурні/не процедурні і універсальні/спеціалізовані
	\item Які традиційні функції обробки типу даних ``рядок'' вам відомі?
	% порівняння рядків, виділення підрядка, конкатенація рядків, тощо
	\item Які два класи задач пов'язаних з синтаксичними структурами програм вам відомі?
	% задачі аналізу та задачі синтезу
	\item Які два типи мовних процесорів вам відомі?
	% транслятори та інтерпретатори
	\item Опишіть структуру транслятора.
	% вхід/лесика/синтаксис/семантика/оптимізація/вихід
	\item Що таке лексема?
	% ланцюжок літер, що має певний зміст
	\item Які два поділи оптимізації ви знаєте? 
	% за пам'яттю/часом і машинно-(не)залежні
\end{enumerate}
