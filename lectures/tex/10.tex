\setcounter{section}{9}

\section{Синтаксичний аналіз на \texorpdfstring{$LL(1)$}{LL1}-гра\-ма\-ти\-ках}

\subsection{Синтаксичний аналіз на основі \texorpdfstring{$LL(1)$}{LL1}-гра\-ма\-тик}

Згідно визначення $LL(1)$-граматики, граматика $G$ буде $LL(1)$ граматикою тоді і тільки тоді, коли для кожного $A$-правила вигляду $A \mapsto \omega_1 \mid \omega_2 \mid \ldots \mid \omega_p$ виконуються умови
\begin{itemize}
	\item $\text{First}_1 (\omega_i) \cap \text{First}_1 (\omega_j) = \varnothing$ для довільних $i \ne j$.
	\item якщо $\omega_i \Rightarrow^\star \varepsilon$ для якогось $i$, то $\text{First}_1(\omega_j) \cap \text{Follow}_1 (A) = \varnothing$ для усіх $j \ne i$.
\end{itemize}

Таблиця $M(a, b)$ де $a \in (N \cup \Sigma \cup \{\varepsilon\})$, $b \in (\Sigma \cup \{\varepsilon\})$ \textit{керування} $LL(1)$-синтаксичним аналізатором визначається наступним чином:
\begin{enumerate}
	\item $M(A, b)$ --- номер правила вигляду $A \mapsto \omega_i$ такого, що
	\begin{equation}
	    \{b\} = \text{First}_1(\omega_i \cdot \text{Follow}_1 (A)).
	\end{equation}
	\item $M(a, a)$ містить інструкцію \verb|pop| для аналізатора яка позначає необхідність перенести символ $a$ з пам'яті аналізатору у поле результату.
	\item $M(\varepsilon, \varepsilon)$ містить інструкцію \verb|accept| для аналізаторп яка позначає що опрацьоване слово необхідно допустити (повернути \verb|true| абощо).
	\item В інших випадках $M(a, b)$ невизначено, чи радше містить інструкцію \verb|reject| для аналізатора яка позначає що опрацьоване слово необхідно недопустити (повернути \verb|false| абощо).
\end{enumerate}

\subsubsection{Приклад}

Розглянемо вже добре відому нам граматику зі схемою
\begin{align*}
	S &\mapsto BA, \\
	A &\mapsto +BA \mid \varepsilon, \\
	B &\mapsto DC, \\
	C &\mapsto \times DC \mid \varepsilon, \\
	D &\mapsto (S) \mid a.
\end{align*}
і пронумеруємо її правила таким чином:
\setcounter{equation}{0}
\begin{align}
	S &\mapsto BA, \\
	A &\mapsto +BA, \\
	A &\mapsto \varepsilon, \\
	B &\mapsto DC, \\
	C &\mapsto \times DC, \\
	C &\mapsto \varepsilon, \\
	D &\mapsto (S), \\
	D &\mapsto a.
\end{align}

Нагадаємо що для цієї граматики
\begin{equation}
    \begin{aligned}
        \text{First}_1 (S) &= \text{First}_1 (B) = \text{First}_1 (D) = \{(, a\}, \\
        \text{First}_1 (A) &= \{+, \varepsilon\}, \\
        \text{First}_1 (A) &= \{\times, \varepsilon\}
    \end{aligned}
\end{equation}

а також 
\begin{equation}
    \begin{aligned}
        \text{Follow}_1 (S) &= \text{Follow}_1 (A) = \{ \varepsilon, )\}, \\
        \text{Follow}_1 (B) &= \text{Follow}_1 (C) = \{+, \varepsilon, )\}, \\
        \text{Follow}_1 (D) &= \{+, \times, \varepsilon, )\}.    
    \end{aligned}
\end{equation}

Знайдемо множини $\text{First}_1(\omega_i \cdot \text{Follow}_1 (A))$ як $\text{First}_1(\omega_i) \oplus_1 \text{Follow}_1 (A))$ використовуючи результати минулої лекції. \medskip

При побудові таблиці $M(a,b)$ керування $LL(1)$-синтаксичним аналізатором достатньо лише побудувати першу її частину, тобто ту яка з $N \times (\Sigma \cup \{\varepsilon\})$, оскільки решта таблиці визначається стандартно:
\begin{table}[H]
	\centering
	\begin{tabular}{|c|c|c|c|c|c|c|c|}
		\hline
		& $a$ & $($ & $)$ & $+$ & $\times$ & $\varepsilon$ \\ \hline
		$S$ & 1 & 1 &  &  &  &   \\ \hline
		$A$ &  &  & 3 & 2 &  & 3 \\ \hline
		$B$ & 4 & 4 &  &  &  &  \\ \hline
		$C$ &  &  & 6 & 6 & 5 & 6 \\ \hline
		$D$ & 8 & 7 &  &  &  &  \\ \hline
	\end{tabular}
\end{table}

\subsubsection{Алгоритм}

Побудуємо $LL(1)$-синтаксичний аналізатор на основі таблиці керування $M(a,b)$:
\begin{enumerate}
	\item Прочитаємо поточну лексему з вхідного файла, у стек магазинного автомата занесемо аксіому $S$.
	\item Загальний крок роботи:
	\begin{itemize}
		\item Якщо на вершині стека знаходиться нетермінал $A_i$, то активізувати рядок таблиці, позначений $A_i$. Елемент
		\begin{equation}
		    M(A_i, \langle\text{поточна лексема}\rangle)
		\end{equation}
		визначає номер правила, права частина якого заміняє $A_i$ на вершині стека.
		\item Якщо на вершині стека лексема $a_i = \langle\text{поточна лексема}\rangle$, то з вершини стека зняти $a_i$ та прочитати нову поточну лексему.
		\item Якщо стек порожній та досягли кінця вхідного файла, то вхідна програма синтаксично вірна.
		\item В інших випадках --- синтаксична помилка.
	\end{itemize}
\end{enumerate}

\subsubsection{Майже \texorpdfstring{$LL(1)$}{LL1}-граматики}

У деяких випадках досить складно (а інколи й принципово неможливо побудувати $LL(1)$-граматику для реальної мови програмування. При цьому $LL(1)$-властивість задовольняється майже для всіх правил --- лише декілька правил створюють конфлікт, але для цих правил задовольняється \textbf{сильна} $LL(2)$-властивість. Тоді таблиця $M(a,b)$ визначається в такий спосіб:
\begin{itemize}
	\item $M(A,b) = \langle\text{номер правила}\rangle$ вигляду $A_i \mapsto \omega_i$, такого, що
	\begin{equation}
	    b \in \text{First}_1(\omega_i \cdot \text{Follow}_1(A))
	\end{equation}
	\item $M(A,b) = \langle\text{ім'я допоміжної програми}\rangle$ за умови, що
	\begin{equation}
	    b \in \text{First}_1(\omega_i \cdot \text{Follow}_1(A)) \cap b \in \text{First}_1(\omega_j \cdot \text{Follow}_1(A)), \quad i \ne j
	\end{equation}
\end{itemize}

Програма, яка виконує додатковий аналіз вхідного ланцюжка, повинна:
\begin{itemize}
	\item прочитати додатково одну лексему;
	\item на основі двох вхідних лексем вибрати необхідне правило або сигналізувати про синтаксичну помилку;
	\item у випадку, коли правило вибрано, необхідно повернути додатково прочитану лексему у вхідний файл.
\end{itemize}

Звичайно, необхідно модифікувати алгоритм $LL(1)$-син\-так\-сич\-но\-го \allowbreak аналізатора. \medskip

При цьому підпрограма аналізу конфліктної ситуації повинна додатково прочитати нову вхідну лексему, далі скориставшись контекстом з двох лексем, визначити номер правила, яке замість нетермінала на вершині стека та повернути додатково прочитану лексему у вхідний файл.

\subsection{Контрольні запитання}

\begin{enumerate}
	\item Які дві умови повинна задовольняти граматика щоб бути $LL(1)$-гра\-ма\-ти\-кою?
	\item Що таке таблиця керування синтаксичного аналізатора на основі $LL(1)$-гра\-ма\-ти\-ки?
	\item Який автомат і яка таблиця використовуються в алгоритмі роботи $LL(1)$-син\-так\-сич\-но\-го аналізатора? % магазинний
	\item Опишіть алгоритм роботи $LL(1)$-синтаксичного аналізатора.
	\item Як необхідно модифікувати таблицю керування для сильної $LL(2)$-гра\-ма\-ти\-ки яка є майже $LL(1)$-гра\-ма\-ти\-кою?
	\item Як необхідно модифікувати алгоритм роботи синтаксичного аналізатора для сильної $LL(2)$-гра\-ма\-ти\-ки яка є майже $LL(1)$-гра\-ма\-ти\-кою?
\end{enumerate}