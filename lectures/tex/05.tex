\setcounter{section}{4}

\section{Регулярні множини і регулярні вирази}

\subsection{Регулярні множини}

Нехай $\Sigma$ --- скінчений алфавіт. \textit{Регулярна множина} в алфавіті $\Sigma$ визначається рекурсивно:
\begin{enumerate}
	\item $\varnothing$ --- пуста множина --- це регулярна множина в алфавіті $\Sigma$;
	\item $\{\varepsilon\}$ --- пусте слово --- регулярна множина в алфавіті $\Sigma$;
	\item $\{a\}$ --- однолітерна множина --- регулярна множина в алфавіті $\Sigma$;
	\item Якщо $P$ та $Q$ --- регулярні множини, то такими є наступні множини:
	\begin{itemize}
		\item $P \cup Q$ (операція об'єднання);
		\item $P \times Q$ (операція конкатенації);
		\item $P^\star = \{\varepsilon\} \cup P \cup P^2 \cup \ldots$ (операція ітерації).
	\end{itemize}
	\item Ніякі інші множини, окрім побудованих на основі 1--4 не є регулярними множинами.
\end{enumerate}

Таким чином, регулярні множини можна побудувати з базових елементів 1--3 шляхом скінченого застосування операцій об'єднання, конкатенації та ітерації.

\subsection{Регулярні вирази}

Регулярні вирази позначають регулярні множини таким чином, що:
\begin{enumerate}
	\item $0$ позначає регулярну множину $\varnothing$;
	\item $\varepsilon$ позначає регулярну множину $\{\varepsilon\}$;
	\item $a$ позначає регулярну множину $\{a\}$;
	\item Якщо $p$ та $q$ позначають відповідно регулярні множини $P$ та $Q$, то
	\begin{itemize}
		\item $p + q$ позначає регулярну множину $P \cup Q$;
		\item $p \cdot q$ позначає регулярну множину $P \times Q$;
		\item $p^\star$ позначає регулярну множину $P^\star$.
	\end{itemize}
	\item Ніякі інші вирази, окрім побудованих на основі 1--4 не є регулярними виразами.
\end{enumerate}

\subsubsection{Алгебра регулярних виразів}

Оскільки ми почали вести мову про вирази, нам зручніше перейти до поняття алгебри регулярних виразів. Для кожної алгебри одним з важливих питань є питання еквівалентних перетворень, які виконуються на основі тотожностей у цій алгебрі. Сформулюємо основні тотожності алгебри регулярних виразів:
\begin{enumerate}
	\item $a + b + c = a + (b + c)$ (ліва асоціативність додавання);
	\item $a + b + c = (a + b) + c$ (права асоціативність додавання);
	\item $a + 0 = 0 + a = a$ ($0$ --- нейтральний елемент за додаванням);
	\item $a \cdot b \cdot c = a \cdot (b \cdot c)$ (ліва асоціативність множення);
	\item $a \cdot b \cdot c = (a \cdot b) \cdot c$ (права асоціативність множення);
	\item $a + b = b + a$ (комутативність додавання);
	\item $a \cdot \varepsilon = \varepsilon \cdot a = a$ ($\varepsilon$ --- нейтральний елемент за множенням);
	\item $a \cdot 0 = 0 \cdot a = 0$ ($0$ --- нульовий елемент за множенням);
	\item $a \cdot (b + c) = a \cdot b + a \cdot c$ (ліва дистрибутивність множення відносно додавання);
	\item $(a + b) \cdot c = a \cdot c + b \cdot c$ (права дистрибутивність множення відносно додавання).
\end{enumerate}

У алгебри регулярних виразів є і некласичні властивості:
\begin{enumerate}
	\item $a + a = a$;
	\item $p + p^\star = p^\star$;
	\item $0^\star = \varepsilon$;
	\item $\varepsilon^\star = \varepsilon$.
\end{enumerate}

\subsubsection{Лінійні рівняння}

За аналогією з класичними алгебрами розглянемо лінійне рівняння в алгебрі регулярних виразів: $X = a \cdot X + b$, де $a$, $b$ --- регулярні вирази. \medskip

Взагалі кажучи це рівняння (в залежності від $a$ та $b$) може мати безліч роз\-в'яз\-ків. \medskip

Серед всіх розв'язків рівняння з регулярними коефіцієнтами виберемо найменший розв'язок $X = a^\star \cdot b$, який назвемо \textit{найменша нерухома точка}. \medskip

Щоб перевірити, що $a^\star \cdot b$ справді розв'язок рівняння в алгебрі регулярних виразів, підставимо його в початкове рівняння та перевіримо тотожність виразів на основі системи тотожних перетворень:
\[ a^\star b = a a^\star b = (a a^\star + \varepsilon) b = (a (\varepsilon + a + a^2 + \ldots ) + \varepsilon) b = (\varepsilon a + a^2 + \ldots) b = a^\star b. \]

\subsubsection{Системи рівнянь}

В алгебрі регулярних виразів також розглядають і системи лінійних рівнянь з регулярними коефіцієнтами:
\begin{equation}
	\left\{
		\begin{aligned}
			X_1 &= a_{11} X_1 + a_{12} X_2 + \ldots + a_{1n} X_n + b_1, \\
			X_2 &= a_{21} X_1 + a_{22} X_2 + \ldots + a_{2n} X_n + b_2, \\
			&\ldots \\
			X_n &= a_{n1} X_1 + a_{n2} X_2 + \ldots + a_{nn} X_n + b_n.
		\end{aligned}
	\right.
\end{equation}

Метод розв'язування системи лінійних рівнянь з регулярними коефіцієнтами нагадує метод виключення Гауса.
\begin{enumerate}
	\item Для $i = \overline{1..n}$, використавши систему тотожних перетворень, записати  $i$-е рівняння у вигляді: $X_i = a X_i + b$, де $a$ --- регулярний вираз в алфавіті $\Sigma$, а $b$ --- регулярний вираз виду \[\beta_0 + \beta_{i+1} X_{i+1} + \beta_{i+2} X_{i+2} + \ldots + \beta_n X_n,\]	де $\beta_k$ ($k = 0, \overline{i+1..n}$) --- регулярні коефіцієнти. Далі, в правих частинах рівнянь зі змінними $X_{i+1}, X_{i+2}, \ldots, X_n$ в лівій частині рівняння підставити замість $X_i$ значення $a^\star b$.

	\item Для $i = \overline{n..1}$ розв'язати $i$-е рівняння яке зараз має вигляд $X_i = a X_i + b$, де $a$, $b$ --- регулярні вирази в алфавіті $\Sigma$, і підставити його розв'язок $a^\star b$	у коефіцієнти рівнянь зі змінними $X_{i-1}, X_{i-2}, \ldots, X_1$.
\end{enumerate}

\textbf{Приклад.} Розв'язати систему $2\times2$:
\begin{equation}
	\left\{
		\begin{aligned}
			X_1 &= a_{11} X_1 + a_{12} X_2 + b_1, \\
			X_2 &= a_{21} X_1 + a_{22} X_2 + b_2.
		\end{aligned}
	\right.
\end{equation}

\textbf{Розв'язок:}
\begin{enumerate}
	\item З першого рівняння
	\begin{equation}
		X_1 = a_{11}^\star (a_{12} X_2 + b_1).
	\end{equation}

	\item Підставляємо у друге рівняння, воно набуває вигляду
	\begin{equation}
		X_2 = a_{21} (a_{11}^\star (a_{12} X_2 + b_1)) + a_{22} X_2 + b_2.
	\end{equation}

	Або, після спрощень:
	\begin{equation}
		X_2 = (a_{21} a_{11}^\star a_{12} + a_{22}) X_2 + (a_{21} a_{11}^\star b_1 + b_2).
	\end{equation}

	Звідки знаходимо:
	\begin{equation}
		X_2 = (a_{21} a_{11}^\star a_{12} + a_{22})^\star (a_{21} a_{11}^\star b_1 + b_2).
	\end{equation}

	\item Підставляємо у вираз для $X_1$:
	\begin{equation}
		X_1 = a_{11}^\star (a_{12} (a_{21} a_{11}^\star a_{12} + a_{22})^\star (a_{21} a_{11}^\star b_1 + b_2) + b_1).
	\end{equation}

	Тут можна розкрити дужки, але зараз у цьому немає потреби.
\end{enumerate}

\subsection{Контрольні запитання}

\begin{enumerate}
	\item Яка множина називається регулярною (в алфавіті $\Sigma$)? % {}, {eps}, {a}, P u Q, P x Q, P*
	\item Як регулярні вирази позначаються регулярні множини? % 0, eps, a, p + q, pq, p*
	\item Які класичні тотожності алгебри регулярних виразів ви знаєте? % комутативність +, існування нейтральних, асоціативність + і x, ...
	\item Які не класичні тотожності алгебри регулярних виразів ви знаєте? % a + a = a, p + p* = p*, 0* = eps* = eps
	\item Який розв'язок лінійного рівняння $X = a \cdot X + b$ називається найменшою нерухомою точкою? % a*b
	\item Доведіть, що згаданий у попередньому рівняння вираз справді є роз\-в'яз\-ком.
	\item Сформулюйте алгоритм Гауса розв'язування систем лінійних регулярних рівнянь.
	\item Розв'яжіть систему $3\times3$:
	\begin{equation}
		\left\{
			\begin{aligned}
				X_1 &= a X_1 + b X_2 + c, \\
				X_2 &= c X_1 + a X_2 + b X_3, \\
				X_3 &= b + c X_2 + a X_3.
			\end{aligned}
		\right.
	\end{equation}
\end{enumerate}
